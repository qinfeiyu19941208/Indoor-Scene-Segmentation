\section{Conclusion}
\label{sec:conclusion}
In this paper, we have introduced two relatively simple but very effiective approaches for indoor scene semantic segmentation.  
%
In order to make full use of unlabeled samples, we put forward a method of label propagating and generate the corresponding PGT.
%
Despite the existence of noisy labels in PGT, the network can still learn more abundant information from PGT.
%
In addition, we also propose a training policy that training a semantic segmentation network using temporal consistecny to constrain the continuity between frames.
%
Experimental results on NYUD-v2 dataset shows that proposed method performs
better than the baseline model. 
%
And the mean accuracy has been significantly improved.

%{\bf Future Work.} As we said in the discussion, the imbalance of sample classes results in the error of segmentation results. 
%
%Hence, our future work is still based on the data itself.
%
%One direction is how to balance sample categories, and the other is how to make more efficient and reasonable use of unlabeled video data.

