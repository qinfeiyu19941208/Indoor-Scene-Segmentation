\begin{abstract}
Indoor semantic segmentation is a fundamental task in scene understanding.
%
Most previous approaches focus on improving performances on dense annotated datasets. 
%
However, there is a great amount of unlabled data that has not yet been used.
%
In addition, by studying the prediction results from previous methods, we observe that the "jumping" often occured between adjacent frames.
%
In this paper, we first propose a simple method to make better use of unlabeled data by propogating annotations to their neighbor frames. 
%
Aiming at the "jumping" problem, we propose a video-based semantic segmentation network that using the interframe optiacal flow to constrain that the prediction of adjcent frames are consistent.
%
Experiments on the indoor dataset show that the proposed method achieves the state-of-the-art performance in some indicators.

\end{abstract}
\begin{keywords}
	Indoor Scene Understanding, Pseudo Ground Truth , Video-based Semantic Segmentation Network
\end{keywords}
