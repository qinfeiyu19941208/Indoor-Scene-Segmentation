\begin{abstract}
Indoor semantic segmentation is a fundamental task in scene understanding.
%
Most previous approaches focus on improving performance on dense-annotated datasets. 
%
However, due to the limited amount of dense-labeled data, the performance of existing networks is greatly limited.
%
But, there are often a large number of unlabeled data has not been used.
%
Hence, in this paper, we proposed two simple but effective methods to make better use of unlabeled data and improve segmentation results by using temporal correlation between unlabeled data.
%
The first is propogating provided annotations to their neighboring frames. 
%
Second, training a semantic segmentation network using the interframe optiacal flow to constrain the temporal consistency.
%
Experiments on the indoor dataset show that the proposed methods achieve the state-of-the-art performance in some indicators.

\end{abstract}
\begin{keywords}
	Indoor Scene Understanding, Pseudo Ground Truth, Temporal Consitency
\end{keywords}
